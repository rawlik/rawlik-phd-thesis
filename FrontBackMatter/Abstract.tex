% Abstract

\pdfbookmark[1]{Abstract}{Abstract} % Bookmark name visible in a PDF viewer

\begingroup
\let\clearpage\relax
\let\cleardoublepage\relax
\let\cleardoublepage\relax

\chapter*{Summary} % Abstract name
% BACKGROUND
Measurements of the electric dipole moment of the neutron (nEDM) provide insight into unanswered questions of contemporary physics, such as ``Why is there more matter than antimatter in the Universe?'' and ``Why does the strong interaction in the Standard Model appear to be fine-tuned?''.
The most sensitive nEDM measurement to date was performed in the Paul Scherrer Institute, Villigen, Switzerland.
This work is concerned with two aspects of this experiment. One is active magnetic field shielding for its successor---n2EDM\@.
The other is a search for a signature of dark matter in its measurements.

Stability of the magnetic field is crucial in measurement of the nEDM\@. The experiment at PSI employed an active magnetic shield, which stabilised the field by detecting  variations in it and counteracting them with a set of large coils around the apparatus.
In the design of an active shield for n2EDM tight spatial constraints excluded the use of known coil geometries, and methods of design thereof.
In this work a new method of magnetic field coil design is presented. It achieves an unprecedentedly large ratio of the fiducial volume to the size of the coil.
Spatial constraints are easily incorporated, as the coils are designed on a predefined grid, which may be shared between multiple coils.
A small-scale active magnetic shield demonstrated the method's designs. Field maps showed the homogeneity to be on a \SI{2}{\percent} level.
Actively shielding homogeneous fields increased the stability 2--30 times down to \SI{0.3}{nT} over times from seconds to hours.
Based on those developments a design for an active shield for the n2EDM experiment is proposed.
%which would could potentially reduce the fluctuations of the magnetic field down to few microteslas.
The n2EDM shield can perform better when tailored to the particular magnetic environment.
To that purpose a device to map magnetic fields in a large volume and in a short time was developed: a mobile tower equipped with magnetic sensors, whose position and orientation was continuously measured with string potentiometers.
A small-scale prototype was tested at LPSC, Grenoble, France, demonstrating a reproducibility in the measurement of \SI{138}{nT} for the field and \SI{3.8}{nT/cm} of its gradient.
A full-scale \SI{8}{m} high tower was used to map the experimental site of n2EDM\@.
The maps can be used to refine the proposed design of the n2EDM active shield.

Dark matter is an estimated \SI{84}{\percent} of the matter content of the Universe, based on astrophysical observations of its gravitational interaction with the visible matter.
Yet, the dark matter constraints remain elusive for particle physics.
An axion is a prominent candidate for a dark matter particle, which is expected to couple to ordinary matter in other ways, besides gravity.
In particular, it may couple to the neutrons in the PSI nEDM experiment via a scalar axion-gluon or a derivative axion-nucleon coupling, causing harmonic oscillations in their spin-precession frequency.
In this work such oscillations were sought.
The search covered periods between \num{1.5} year and \SI{300}{\second}, which corresponds to axion masses $\sim \SI{e-22}{\electronvolt}$ -- \SI{2e-17}{\electronvolt}.
The null result put the first laboratory constraints on the axion coupling to gluons and improved ones on the axion-nucleon coupling by a factor of 40.

\enlargethispage{2\baselineskip}

\endgroup			

\vfill
