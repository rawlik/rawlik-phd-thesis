% Abstract

\pdfbookmark[1]{Abstract}{Abstract} % Bookmark name visible in a PDF viewer

\begingroup
\let\clearpage\relax
\let\cleardoublepage\relax
\let\cleardoublepage\relax

\chapter*{Zusammenfassung} % Abstract name

\newcommand{\jkfootnote}[1]{\footnote{#1}}

\begin{otherlanguage}{german}

Messungen des elektrischen Dipolmoments des Neutrons (nEDM) bringen Einsicht in offene Fragen der heutigen Physik wie zum Beispiel: \glqq{}Warum gibt es mehr Materie als Anti-Materie im Universum?\grqq{} und \glqq{}Warum scheint die starke Wechselwirkung im Standardmodel eine unnatürliche Feinabstimmung zu haben?\grqq

Die Messung des nEDM mit der bislang grössten Empfindlichkeit wurde am Paul Scherrer Institut (PSI) in Villigen, Schweiz durchgeführt.
Diese Arbeit behandelt zwei Aspekte dieses Experimentes. Eine ist die aktive Magnetfeldabschirmung  für das Nachfolgeexperiment n2EDM\@.
Die andere ist die Suche nach Hinweisen auf Dunkle Materie in den Messdaten.

Stabilität des magnetischen Feldes ist entscheidend bei der Messung des nEDM\@.
Das Experiment am PSI verwendet eine aktive magnetische Schirmung, die das Feld stabilisiert, indem es Veränderungen detektiert und diesen mithilfe einiger grosser Spulen um den Apparat entgegenwirkt.

Bei der Konstruktion\jkfootnote{maybe need better} der aktiven Abschirmung für n2EDM verhinderten Platzmangel\jkfootnote{räumliche Einschränkungen} die Verwendung bekannter Spulengeometrieen und übliche Methoden solche zu erzeugen.

In dieser Arbeit wird eine neue Methode zur Erzeugung einer Spulengeometrie vorgestellt. Dieser erreicht ein beispielloses Grössenverhältnis zwischen der Spulen und dem geschirmten Bereich.
Platzeinschränkungen können sehr einfach berücksichtigt werden und sie Spulen werden auf einem vordefinierten Gitter angeordnet. Mehrere Spulen können das gleiche Gitter verwenden.

Ein verkleinertes Model der Magnetfeldabschirmung wurde zur Demonstration gebaut. Feldkartografierungen zeigten eine Homogenität auf dem \SI{2}{\percent} Niveau.

Das aktive Schirmen von homogenen Feldern verbessert deren Stabilität 2--30-fach auf \SI{0.3}{nT} über den Zeitbereich von Sekunden zu Stunden.
Anhand dieser Erkenntnisse wird ein Design für die aktive Magnetfeldabschirmung für n2EDM vorgeschlagen.

Die Schirmung für n2EDM kann besser sein, wenn sie auf die dortige magnetische Umgebung angepasst wird.
Dafür wurde ein Gerät zum Kartographieren des Magnetfeldes im einem grossen Volumen entwickelt: ein fahrbarer Turm mit Magnetfeldsensoren dessen Position und Ausrichtung kontinuierlich mit Seilzugsensoren gemessen wird.
Ein kleiner Prototyp wurde am LPSC in Grenoble, Frankreich getestet. Eine Reproduzierbarkeit der Messergebnisse von \SI{138}{nT} für das Feld und \SI{3.8}{nT/cm} seiner Gradienten wurde demonstriert.
Der Turm in voller Grösse  (\SI{8}{m}) wurde verwendet um das Experiment-Areal für n2EDM zu vermessen.
Die Feldkarten können verwendet werden um das vorgeschlagen Design für die aktive Schirmung von n2EDM zu optimieren.


Die Materie des Universums besteht zu \SI{84}{\percent} aus Dunkler Materie, wie aus astrophsykalischen\jkfootnote{Astronomischen ?} Beobachtungen von sichtbarer Materie, die mit dunkler Materie nur per Schwerkraft interagieren kann, abgeschätz wird.\jkfootnote{aahhhhhhh}
Allerdings kann Dunkle Materie bislang nicht von der Teilchenphysik erfasst werden.
Das Axion ist ein beliebter Kandidat für ein Konstituententeilchen der Dunklen Materie.\jkfootnote{difficult}
Es wird erwartet, dass es an gewöhnlicher Materie auch durch andere Kräfte als die Gravitation koppelt.
Insbesondere könnte es an die Neutronen im nEDM-Experiment am PSI mit einer skalaren Axion-Glonen odervektorartigen Axion-Nukleonen Wechselwirkung koppeln und so harmonische Oszillationen derer Larmorfrequenzen erzeugen.
In dieser Arbeit wurde nach einer solchen Oszillation gesucht.
Die Suche deckte Perioden von \num{1.5} bis \SI{300}{\second} ab, was einer Masse des Axions von $\sim \SI{e-22}{\electronvolt}$ bis \SI{2e-17}{\electronvolt} entspricht.
Das Nullergebnis stellt die erste experimentelle Obergrenze der Kopplung von Axionen zu Gluonen dar, und verbessert die Grenze der Kopplung von Axion zu Nukleon auf das 40-fache.

\end{otherlanguage}

\endgroup

\vfill
