% Acknowledgements

\pdfbookmark[1]{Curriculum vitae}{Curriculum vitae} % Bookmark name visible in a PDF viewer


%----------------------------------------------------------------------------------------

\manualmark
\markboth{\spacedlowsmallcaps{\bibname}}{\spacedlowsmallcaps{\bibname}} 
\refstepcounter{dummy}

\addtocontents{toc}{\protect\vspace{\beforebibskip}} % Place the bibliography slightly 
% below the rest of the document content in the table of contents
% Add Acknowledgements manually to toc, so that it does not get a letter as it were a part of an appendix
\addcontentsline{toc}{chapter}{\tocEntry{Curriculum vitae}}

\begingroup

\let\clearpage\relax
\let\cleardoublepage\relax
\let\cleardoublepage\relax

\chapter*{Curriculum vitae} % Acknowledgements section text

Michał Rawlik, born on July 18\textsuperscript{th} 1990 in Bytom, Poland. Nationality: Polish. 

\bigskip

\section*{Education}
\paragraph{ETH Zürich} 2014--2018, \emph{Zürich, Switzerland}\\
Doctorate in the group of Prof.\ Klaus Kirch in the Institute of Particle Physics and Astrophysics.

\paragraph{Jagiellonian University} 2012--2014, \emph{Kraków, Poland}\\
Graduate studies. Master's Thesis \emph{FID signal analysis and new DAQ system in the nEDM experiment} in the group of Prof.\ Kazimierz Bodek in the
Faculty of Physics, Astronomy and Applied Computer Science.

\paragraph{Jagiellonian University} 2009--2012, \emph{Kraków, Poland}\\
Undergraduate studies. Bachelors's Thesis \emph{Tuning neutron RF-pulse frequency in the nEDM experiment} in the group of Prof.\ Kazimierz Bodek in the
Faculty of Physics, Astronomy and Applied Computer Science.

\paragraph{High School No. 1} 2006--2009, \emph{Gliwice, Poland}

\paragraph{Secondary School ETE} 2003--2006, \emph{Gliwice, Poland}

\paragraph{Primary School No. 28} 1997--2003, \emph{Gliwice, Poland}




% \section*{Publications}
% \emph{Search for axion-like dark matter through nuclear spin precession in electric and magnetic fields}, Phys. Rev. X \textbf{7}, 041034 (2017)

% \emphA simple method of coil design	arxiv.org/abs/1709.04681 (2017)
% Ultracold neutron detection with 6Li-doped glass scintillators	Eur. Phys. J. A 52: 326 (2016)
% Gravitational Depolarization of Ultracold Neutrons:	Phys. Rev. D 92, 052008 (2015)
% Comparison with Data
% A Revised Experimental Upper Limit on the Electric	Phys. Rev. D 92, 092003 (2015)
% Dipole Moment of the Neutron
% A highly stable atomic vector magnetometer	Opt. Express 23, 22108-22115 (2015)
% based on free spin precession
% Observation of Gravitationally Induced Vertical Striation	Phys. Rev. Lett. 115, 162502 (2015)
% of Polarized Ultracold Neutrons by Spin-Echo Spectroscopy
% A device for simultaneous spin analysis of ultracold neutrons	Eur. Phys. J. A 51, 143 (2015)


\bigskip

\section*{Internships}
\paragraph{Department of Cognitive Neuroscience and Neuroergonomics UJ} 2013, \emph{Kraków, Poland}
\paragraph{Paul Scherrer Institute} 2012 \& 2013, \emph{Villigen, Switzerland}
\paragraph{Forschungszentrum Jülich} 2011, \emph{Jülich, Germany}
\paragraph{Institute of Nuclear Physics PAN} 2011, \emph{Kraków, Poland}



\endgroup
