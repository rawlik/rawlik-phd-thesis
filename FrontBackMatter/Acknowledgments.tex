% Acknowledgements

\pdfbookmark[1]{Acknowledgements}{Acknowledgements} % Bookmark name visible in a PDF viewer


%----------------------------------------------------------------------------------------

\manualmark
\markboth{\spacedlowsmallcaps{Acknowledgements}}{\spacedlowsmallcaps{Acknowledgements}} 
\refstepcounter{dummy}

\addtocontents{toc}{\protect\vspace{\beforebibskip}} % Place the bibliography slightly 
% below the rest of the document content in the table of contents
% Add Acknowledgements manually to toc, so that it does not get a letter as it were a part of an appendix
\addcontentsline{toc}{chapter}{\tocEntry{Acknowledgements}}

\begingroup

\let\clearpage\relax
\let\cleardoublepage\relax
\let\cleardoublepage\relax

\chapter*{Acknowledgements} % Acknowledgements section text

% I have many people to thank. Some helped me directly with the work,  some were and inspiration and many created the atmosphere and supported me during the period of my doctoral studies.

% some where inspiring, some were important during this period.

I would like to thank Klaus Kirch for giving me an opportunity to pursue a PhD in his group, for his strategic guidance and for sharing his expertise.
I thank Florian Piegsa and Jochen Krempel for their day-to-day advice.
The search for the axions would never happen without the theoretical support of Victor Flambaum, Malcolm Fairbairn, Yevgeny Stadnik and Doddy (David J.E. Marsh).
I greatly enjoyed collaborating with Nick Ayres on the axion analysis.
I would never be able to work my way through the data of the nEDM experiment without Elise Wursten's help.
I am grateful for insightful discussions about magnetic field coils I had with Gilles Quéméner and Chris Crawford.
Towards the end of my work I closely collaborated with Solange Emmenegger. Thank you for all the laughs.
Furthermore, I am grateful to the whole nEDM collaboration.
The ideas presented in this thesis were discussed over and over again in the friendly and inspiring atmosphere of our meetings.
I would like thank, in particular,
Philipp Schmidt-Wellenburg,
Georg Bison,
Bernhard Lauss,
Dieter Ries,
Sybille Komposch,
Chris Abel,
Nicolas Hild,
Vira Bondar,
Małgorzata Kasprzak,
Hans-Christian Koch,
Guillaume Pignol,
% Stephanie Roccia,
% Gilles Ban,
% Manfred Daum,
% Peter Geltenbort,
% Clark Griffith,
% Zoran D. Grujić,
Philip Harris
% Dominique Rebreyend,
% Yoann Kerma\"{i}dic,
% Peter Koss,
% Thomas Lefort,
% Yves Lemière,
% Prajwal Mohanmurthy,
% Aliko Mtchedlishvili,
% Matthew Musgrave.
% Nathal Severijns,
% Antoine Weis and
% Geza Zsigmond,
and all the other members of the collaboration.
I am grateful to the group at the Jagiellonian University in Cracow, Poland, where I did my Master's project.
My thanks goes to
Kazimierz Bodek,
Jacek Zejma,
Dagmara Rozpędzik,
Grzegorz Wyszyński and
Adam Kozela
for introducing me not only to the nEDM community, but also to the trade of experimental physics itself.
During my PhD I spent most of my working time at ETH, where I enjoyed many discussions during breaks for coffee, tea and juggling.
My thanks goes to
Andreas Eggenberger,
Aldo Antognini,
Ivana Belo\v{s}ević,
Kim Siang Khaw,
Karsten Schuhmann,
Gunther Wichmann,
Mirosław Marszałek,
Zachary Hodge and
Manuel Zeyen.
I benefited from an excellent technical support of Michi Meier and Fritz Burri at PSI, and Bruno Zehr at ETH\@.
I did a lot of the laboratory work together with apprentices who, despite their young age, amazed me with their skills.
Thank you,
Mona Hänni,
Mirco Dill,
Moritz Stöckli and
Cyrill Strässle.
I also greatly enjoyed working with students.
I thank
Tizian Bluntschli,
Nick Schwegler,
Gianluca Janka,
Avraam Chatzimichailidis,
Daria Cegiełka,
Enrico del Re and
Hanno Bertle for putting heart in their work.
I would be completely lost without administrative help from Anita van Loon, Rosa Bächli,
Gaby Amstutz, Bettina Lareida and Caroline Keufer-Platz.
For all their support I thank my parents Gabriela and Grzegorz.
Most importantly, I owe a great many thanks to my wonderful wife Agata and our darling son Jędrzej.

\endgroup


\newpage
\section*{Tools}
During my work I used many open-source tools: matplotlib~\cite{Hunter2007}, Python, Julia~\cite{julia}, numpy, scipy~\cite{scipy}, jupyter~\cite{jupyter}, pyqtgraph, ipython~\cite{ipython}, pandas~\cite{pandas}, inkscape, Ipe and countless libraries for Julia and Python.
I am grateful to the countless contributors to these remarkable pieces of software. This thesis was typeset using the \emph{classicthesis} template by André Miede.
