% Acknowledgements

\pdfbookmark[1]{Acknowledgements}{Acknowledgements} % Bookmark name visible in a PDF viewer


%----------------------------------------------------------------------------------------

\manualmark
\markboth{\spacedlowsmallcaps{\bibname}}{\spacedlowsmallcaps{\bibname}} 
\refstepcounter{dummy}

\addtocontents{toc}{\protect\vspace{\beforebibskip}} % Place the bibliography slightly 
% below the rest of the document content in the table of contents
% Add Acknowledgements manually to toc, so that it does not get a letter as it were a part of an appendix
\addcontentsline{toc}{chapter}{\tocEntry{Acknowledgements}}

\begingroup

\let\clearpage\relax
\let\cleardoublepage\relax
\let\cleardoublepage\relax

\chapter*{Acknowledgements} % Acknowledgements section text

% I have many people to thank. Some helped me directly with the work,  some were and inspiration and many created the atmosphere and supported me during the period of my doctoral studies.

% some where inspiring, some were important during this period.

I would like to thank Klaus Kirch for giving me an opportunity to pursue a PhD in his group, for his strategic advice and for sharing his expertise.
I thank Florian Piegsa and Jochen Krempel their day-to-day guidance.
I am grateful for insightful discussions about magnetic field coils I had with Gilles Quéméner and Chris Crawford. The search for the axions would never happen without Yevgeny Stadnik, Victor Flambaum, Doddy (David J.E. Marsh) and Malcolm Fairbairn. Collaborating with Nick Ayres on the axion analysis was a wonderful experience. I would never be able to work my way through the data of the nEDM experiment without Elise Wursten's help. Towards the end of my work I closely collaborated with Solange Emmenegger. It was objectively awesome. Thank you for all the laughs and cries. I am grateful to the whole nEDM collaboration. During collaboration meeting the ideas presented in this thesis were discussed over and over again, always in a friendly atmosphere and at PSI I could always count on help. I would like to mention
Philipp Schmidt-Wellenburg,
Georg Bison,
Dieter Ries,
Chris Abel,
Sybille Komposch,
Guillaume Pignol,
Bernhard Lauss,
% Stephanie Roccia,
% Gilles Ban,
Vira Bondar,
% Manfred Daum,
% Peter Geltenbort,
% Clark Griffith,
% Zoran D. Grujić,
Philip Harris,
% Dominique Rebreyend,
Nicolas Hild,
Małgorzata Kasprzak,
% Yoann Kerma\"{i}dic,
Hans-Christian Koch,
Peter Koss and
% Thomas Lefort,
% Yves Lemière,
Prajwal Mohanmurthy.
% Aliko Mtchedlishvili,
% Matthew Musgrave.
% Nathal Severijns,
% Antoine Weis and 
% Geza Zsigmond,
I am grateful to the group at the Jagiellonian University in Cracow, Poland, where I did my Masters' Thesis. My thanks goes to
Kazimierz Bodek,
Jacek Zejma,
Dagmara Rozpędzik,
Grzegorz Wyszyński and
Adam Kozela
for introducing me not only to the nEDM community, but also to the trade of science itself. During my PhD I spent most of my working time at ETH, where I enjoyed great many discussions during breaks for coffee, tea or juggling. My thanks goes to
Aldo Antognini,
Ivana Belo\v{s}ević,
Andreas Eggenberger,
Mirosław Marszałek,
Manuel Zeyen,
Laura Sinkunaite,
Karsten Schuhmann,
Gunther Wichmann and
Kim Siang Khaw.
I benefited from excellent technical support of Fritz Burri and Michi Meier at PSI, and Bruno Zehr at ETH\@. A lot of the laboratory work I did together with apprentices who, despite their young age, amazed me with their skills. Thank you,
Mona Hänni,
Mirco Dill,
Moritz Stöckli and
Cyrill Strässle.
I also greatly enjoyed working with students. I thank
Tizian Bluntschli,
Nick Schwegler,
Gianluca Janka,
Avraam Chatzimichailidis,
Daria Cegiełka,
Enrico del Re and
Hanno Bertle for putting heart in their work.
I would be completely lost without administrative help from Anita van Loon, Rosa Bächli
Gaby Amstutz, Bettina Lareida and Caroline Keufer-Platz. For all their support I thank my parents: Gabriela and Grzegorz. The love of my wife Agata and our son Jędrzej. 




\section*{Tools}
During my work I used many open-source tools: matplotlib~\cite{Hunter2007}, python, julia~\cite{julia}, numpy, scipy~\cite{scipy}, jupyter~\cite{jupyter}, pyqtgraph, ipython~\cite{ipython}, pandas~\cite{pandas}.
I am grateful to the countless contributors to these remarkable pieces of software. This thesis was typeset using the \emph{classicthesis} template by André Miede.


\endgroup