\chapter{The nEDM active magnetic shielding}

\label{ch:nedm_sfc}


\section{The principle of an active magnetic shielding}
\note{This is the first time the idea of an active comensation system is introduced! Need to be slow here.}
Describe it only in general! That the experiment is in the centre, there are magnetic field sensors, normally fluxgates, and coils around. Do not set the feedback algorithm at this point, will explain later. Everything here is valid for both Bea's and my systems.

Say that there is a volume surrounded by magnetic field coils. There are sensors in the middle of the volume. In each step the sensors detect a change, he readouts are then processed to calculate a current response to the change detected. The signals is amplified and put into the coils.

The advantages are price and spatial constraints (as opposed to mu-mental) they do not enclose the volume completely. Mostly, they are effective in the different frequency regime. The shielding factor of mu-metal degrades by as much as two orders of magnitude for frequencies slower than 1-10Hz \cite{Brake1991}. At the same time active systems perform best at kHz all the way to DC. The combination of the two provides a stable magnetic field over a wide range of frequencies.

Systems like \cite{Kobayashi2012}, \cite{Spemann2003}, \cite{RetaHernandez1998}, \cite{Voigt2013}, \cite{Brake1991} were developed. And, notably the nEDM system~\cite{Afach2014}, about to be discussed in this chapter.


\section{The nEDM SFC system}
Here shortly about the history and Bea's thesis. The next section describes Say that the system was in operation and I have taken over the maintanence. Set my role!

Aside from maintenance, understanding (data analysis). No improvements! Working system, after all, huh?

Say that it was published and cite it \cite{Afach2014}.




\section{Introduction}
\cite{Franke2013}

\subsection{Motivation}

\subsection{The nEDM SFC system}

\subsection{SFC matrix}
\begin{figure}[bth]
  \myfloatalign
  \includegraphics[width=.8\linewidth]{gfx/nEDM_SFC/nEDM_SFC_matrix}
  \caption
  [TODO]
  {The SFC matrix measured by Franke on 2012-11-07 (see \cite{Franke2013}) in the thermohouse coordinates.}
  \label{fig:nEDM_SFC_matrix}
\end{figure}

\subsection{Fluxgates}


\section{Influence of the SC magnet}
The ultracold neutrons are polarised by passing through a superconducting magnet. The length of the path of the neutrons to the experiment should be small to minimise transport losses. For this reason the experiment is located close to the exit of the UCN beam--port and the magnet is fit in between.

\begin{figure}[bth]
  \myfloatalign
  \includegraphics[width=.7\linewidth]{gfx/nEDM_SFC/SCM_magn_map.pdf}
  \caption
  [TODO]
  {Map of the magnetic field of the superconducting magnet. Courtesy of Dr. Geza Zsigmond}
  \label{fig:nEDM_SFC_SC_magnet_map}
\end{figure}

\begin{figure}[bth]
  \myfloatalign
  \subfloat[Displaying the calculated back external field, the movable yellow cursor chooses the data to be displayed in the 3D view.]{
    \label{fig:nEDM_SFC_SC_magnet_influence_top_view}
    \includegraphics[width=.45\linewidth]{gfx/nEDM_SFC/SC_magnet_field_top_view.pdf}}
  \quad
  \subfloat[Displaying fields at two different times simultaneously, yellow and blue.]{
    \label{fig:nEDM_SFC_SC_magnet_influence_front_view}
    \includegraphics[width=.45\linewidth]{gfx/nEDM_SFC/SC_magnet_field_front_view.pdf}}
  \caption{Magnetic field of the SC magnet seen by the SFC. Taken as difference of runs...}
\end{figure}

\begin{figure}[bth]
  \myfloatalign
  \includegraphics[width=.7\linewidth]{gfx/nEDM_SFC/SC_magnet_optimal_coil_on_wall.png}
  \caption
  [TODO]
  {Map of the field of a potential correcting coil}
  \label{fig:nEDM_SFC_SC_magnet_optimal_coil}
\end{figure}

\begin{figure}[bth]
  \myfloatalign
  \subfloat[Displaying the calculated back external field, the movable yellow cursor chooses the data to be displayed in the 3D view.]{
    \label{fig:nEDM_SFC_SC_magnet_influence_top_view2}
    \includegraphics[width=.45\linewidth]{gfx/nEDM_SFC/SC_magnet_improvement_top_view.pdf}}
  \quad
  \subfloat[Displaying fields at two different times simultaneously, yellow and blue.]{
    \label{fig:nEDM_SFC_SC_magnet_influence_front_view2}
    \includegraphics[width=.45\linewidth]{gfx/nEDM_SFC/SC_magnet_improvement_front_view.pdf}}
  \caption{Magnetic field of the SC magnet seen by the SFC. Taken as difference of runs...}
\end{figure}

The improvement in the range of the field is 40-70\% (In different simulated runs, 21.7 to 12.6 (42\%), 22.2 to 12.9 (42\%), 19.3 to 5.8 (70\%), and 35.1 to 18.6 (57\%)), unit microteslas.


\section{Magnetic field during sparks in HV}




\section{Possible mistake in implementation}
Need to provide screenshots of the codehere.

Work of Nick Schwegler.


\section{The spectrum of the SFC matrix}
\label{sec:nedm_sfc_matrix}
The SFC matrix used during the data taking of the nEDM experiment (2014, 15 and 16) is the one measured by Franke \cite{Franke2013}. The matrix not only needed to be inverted, but also additionally regularised. While it has been thoroughly discussed how to do the regularisation, the question of why was it needed at all was neither posed nor answered.

Let us elaborate on regularisation. The SFC matrix represents coefficients in a set of linear equations that need to be solved in order to determine the best currents to apply for a given goal field. As the set of equation is over--determined, the best solution is found by the least--squares, which is exactly equivalent to calculating the pseudoinverse matrix. A solution can be found reliably if the system is well--defined, i.e. the solution ,,dip'' is steep in every direction in the parameter space. If the ,,dip'' becomes a valley in some directions, the solution is not globally well--defined, although it still may be defied up to a parameter (the one pointing in the direction of the valley). We then speak of an ill--defined set of equations, or an ill--defined matrix. Regularisation helps ill--defined problem to become better, at the cost of the least--squares in the solution.


A real matrix $\mathbb{M}$ may be decomposed into $\mathbb{M} = \mathbb{U} \mathbb{S} \mathbb{V}^T$, where $\mathbb{U}$ and $\mathbb{V}$ are unitary, and $\mathbb{S}$ is diagonal. This is called the \emph{Singular Value Decomposition} (SVD). The singular values lie on the diagonal of $\mathbb{S}$, which is call the \emph{spectrum}. The spectrum of the SFC matrix is of uttermost importance. Pseudoinverting a matrix with small singular values is an ill--posed problem similar, as inverting small numbers.

One defines the \emph{condition number} of a matrix as the ratio of extreme values of its spectrum. For a matrix with a flat spectrum, all singular values equal, the condition number is 1 and the set of linear equations this matrix represents is well defined. The more they differ, the higher the condition number and the worse defined the problem is. The effect in the solution is that noise in the original matrix becomes amplified by the condition number in the pseudoinverse.

Figure~\ref{fig:nEDM_SFC_svd} presents the spectrum of the nEDM SFC matrix. First thing to note is that the condition number is $9.6 / 0.51 = 18.2$. This is a factor of 20 amplification in noise and it clearly explains why regularisation was necessary.

It is interesting to ask why. A very small singular value means that there exists a coil, or a combination thereof, which has almost 20 times smaller influence on the field then others. In figure~\ref{fig:nEDM_SFC_svd} columns are the singular values with their corresponding coil--vectors. The first three, starting from the left, are easiest to interpret. Each of them is a pair of coils configured as Helmholtz--pair, with roughly the same current, producing a homogeneous field in each of the spatial directions. The smaller singular value, or the effect on the field, in the Y direction is explained by the fact that this is the longitudinal axis of the μ-metal cylinder. The last one has also a clear interpretation -- it is all pairs configured as anti--Helmholtz, with currents flowing in the opposite directions. The magnetic field that it creates is a very complicated, high-order field. The fact, that this combination has so little influence means, that it hardly changes any solution for currents when added upon it. It spans a valley in the parameter space in the least--squares problem.

\begin{figure}[bth]
  \myfloatalign
  \includegraphics[width=.7\linewidth]{gfx/nEDM_SFC/coil-singular_vectors_of_the_nEDM_SFC_matrix}
  \caption
  [TODO]
  {The coil-singular values of the SFC matrix. Columns correspond to singular combinations of the coils. For each column the corresponding singular value is indicated. See text for details.}
  \label{fig:nEDM_SFC_svd}
\end{figure}

It is to note that the SFC matrix is defined solely by the configuration of the coils and sensors. It follows that care has to be taken already at the design stage to create a system that will be well defined.


\section{Optimising the number of windings in the coils}
Put the plots of the distributions of the currents, juxtapose them with the
ranges of the power supplies.


\section{Visualisation of the data}
The data of the SFC system are difficult to visualise. Mostly because the magnetic field is a three dimensional field in a three dimensional space. It is much easier to explore if they are visualised interactively. Together with a student, Nils Ebeling, we created a visualisation.

Possibility to show the calculated external field!

The program is cross--platform (Windows, OSX, Linux), fully based on open--source softwore. It is written in python and uses the following libraries: Qt4 (cross-platform GUI), pyqtgraph (fast interactive plotting), VTK (3D graphics) and SciPy (data handling).

\begin{figure}[bth]
  \myfloatalign
  \subfloat[Displaying the calculated back external field, the movable yellow cursor chooses the data to be displayed in the 3D view.]{
    \label{fig:nEDM_SFC_visualisation1}
    \includegraphics[width=.45\linewidth]{gfx/nEDM_SFC/visualisation/visualisation1}}
  \quad
  \subfloat[Displaying fields at two different times simultaneously, yellow and blue.]{
    \label{fig:nEDM_SFC_visualisation2}
    \includegraphics[width=.45\linewidth]{gfx/nEDM_SFC/visualisation/visualisation2}}
  \\
  \subfloat[Displaying a range of data, marked with the yellow cursor, relative to the point marked with the blue cursor. The calculated back external magnetic field is displayed.]{
    \label{fig:nEDM_SFC_visualisation3}
    \includegraphics[width=.45\linewidth]{gfx/nEDM_SFC/visualisation/visualisation3}}
  \quad
  \subfloat[Displaying the difference between the external field in two points in geometrical form.]{
    \label{fig:nEDM_SFC_visualisation4}
    \includegraphics[width=.45\linewidth]{gfx/nEDM_SFC/visualisation/visualisation4}}
  \caption{Different functionalities of the nEDM SFC data visualisation tool.}
\end{figure}

Write here about the program and the tricks used (subtract the value under the cursor) and so on. Also about the 3D representation. Here show how does the field of SULTAN look like, in time and in 3D.

Work of Nils Ebeling.


\section{Performance during SULTAN ramps}
Describe the fancy algorithm and show the results of the data analysis.

\begin{figure}[bth]
  \myfloatalign
  \subfloat[Displaying the calculated back external field, the movable yellow cursor chooses the data to be displayed in the 3D view.]{
    \label{fig:nEDM_SFC_SC_magnet_influence_top_view}
    \includegraphics[width=.45\linewidth]{gfx/nEDM_SFC/field_seen_by_Hg_2015-2016}}
  \quad
  \subfloat[Displaying fields at two different times simultaneously, yellow and blue.]{
    \label{fig:nEDM_SFC_SC_magnet_influence_front_view}
    \includegraphics[width=.45\linewidth]{gfx/nEDM_SFC/gradient_seen_by_Cs_2015-2016}}
  \caption{Magnetic field of the SC magnet seen by the SFC. Taken as difference of runs...}
\end{figure}



\section{Remote magnetometers}
Write the documentation, show a SULTAN ramp as seen by the magnetometers, boast a bit about the automatic deployment, timing accuracy and so on.

Work of Gianluca Janka.
