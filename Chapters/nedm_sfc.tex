\chapter{The nEDM active magnetic shielding}

\label{ch:nedm_sfc}


\section{The principle of an active magnetic shielding}
Passive methods of shielding the magnetic field, like $\upmu$-metal shields, rely on magnetic properties of materials. In contrast, in active methods the disturbances are first detected, processed and then counteracted. It is not unlike the recently popular active-noise-cancellation headphones. Standard ones provide only passive damping of the ambient noise by covering the ears. Active headphones additionally feature microphones that resolve the profile of incoming sound, which is then inverted and emitted from the speakers. The two waveforms, when overlaid, cancel.

\begin{figure}
  \centering
  \includegraphics[width=0.7\linewidth]{gfx/nEDM_SFC/SFCplain.pdf}
  \caption{The nEDM-at-PSI active magnetic field compensation system. The experiment, in violet, is in the middle, with nine magnetic field sensors, depicted as green circles, attached to it. Everything is encircled by six coils (orange). Adapted from~\cite{Franke2013}.}
  \label{fig:sfc-scheme}
\end{figure}

Active magnetic field compensation systems follow the same principle. A volume, often called the fiducial volume, is encircled by coils, additionally to having magnetic field sensors distributed in it. It is schematically presented in Fig.\,\ref{fig:sfc-scheme}, taking the nEDM-at-PSI's system as the example: the fiducial volume is the violet structure in the middle, encircled by green dots depicting the magnetic field sensors. Around it there are three perpendicular Helmholtz-like pairs of coils, depicted in shades of orange. Variations of the magnetic field are detected by the sensors, an appropriate counteraction is calculated and sent into the coils.

Active shields do not substitute passive ones. The shielding factor
\marginpar{Shielding factor, measured in \si{\decibel} is the ratio of the power of the magnetic field as if it were not compensated to the compensated one.} of passive shields degrades by as much as \SI{40}{\decibel} (two orders of magnitude in amplitude) for frequencies slower than \SIrange[range-phrase = --]{1}{10}{\hertz}~\cite{Brake1991}. At the same time the active systems perform best at DC and reach up to the kilohertz regime. The combination of the two provides a stable magnetic field over the whole range of frequencies~\cite{Brake1991,Kelha1982,Voigt2013}.

Since 1980s numerous active magnetic field compensation systems have been built \cite{Kelha1982,Brake1991,Spemann2003,Brys2005,Kobayashi2012,Voigt2013,Afach2014}. The applications span from ion beams, through biomagnetism, all the way to nEDM measurements. The system built for the nEDM-at-PSI project is the focus of this chapter.

% \note{Don't discuss it here: In the overlapping range of frequencies, where the two kinds of magnetic shielding could compete, the two advantages of the active shielding would certainly be the cost and accessibility of the enclosed volume.}



\section{The nEDM-at-PSI SFC system}
\marginpar{There was an inside joke that SFC really stands for SULTAN Field Compensation, SULTAN having been the magnet with by far the strongest influence on the magnetic field.}
The construction of nEDM-at-PSI active magnetic field compensation system, called the SFC (Surrounding Field Compensation) system,
was a part of Beatrice Franke's PhD thesis~\cite{Franke2013}, later also published~\cite{Afach2014}. The author of this work was responsible for maintaining the system, with the goal of gaining enough of understanding and experience to design a next generation system for the n2EDM experiment. In the beginning of this chapter the most important information about the SFC system is presented. Then a few new insights into this system are discussed. This serves as an introduction to the original developments, to which the remaining three sections of this part are dedicated.

The distinct feature of the PSI's SFC system was the use of the feedback matrix. Consider a single coil of any shape in air. For every point in space and for every spatial component the magnetic field is directly proportional to the current in the coil. The SFC system had six coils and measures the field in nine points, three spatial components in each: 27 channels. In total there were $6 \times 27 = 162$ proportionality constants, which are gathered in the feedback matrix $M$. The matrix $M$ is a property of the active compensation system, more precisely of the geometry of the coils and sensors. Ferromagnetic materials, in particular $\upmu$-metal, distort the field created by the coils and thereby the feedback matrix. Moreover, they make the system non-linear, which we are going to, for now, neglect.

% \note{Maybe some nice formula here for the matrix elements?}

\begin{figure}
  \centering
  \includegraphics[width=.8\linewidth]{gfx/nEDM_SFC/nEDM_SFC_matrix}
  \caption{The SFC matrix measured by Franke on 2012-11-07~\cite{Franke2013}). For each coil (rows) and each channel of the magnetic field measurement (columns) the proportionality constant between the current in the coil and the field measured by the sensor is depicted. The coils and the sensors are labelled as in Fig.\,\ref{fig:sfc-scheme}.}
  \label{fig:nEDM_SFC_matrix}
\end{figure}

\marginpar{The matrix was first proposed by~\cite{RetaHernandez1998}. As their proposed system did not include $\mu$-metal, they could calculate the matrix analytically.}
The following procedure was used to measure the matrix. A current in one coil was scanned over the whole available range, while the field change was measured with the sensors. Then for each sensor, and each spatial component, a linear regression was performed. The slope was the proportionality constant (the matrix $M$ element). The procedure was repeated of all coils. The matrix measured by Franke is depicted in Fig.\,\ref{fig:nEDM_SFC_matrix}.

There is a fundamental meaning behind the matrix $M$. The currents in the six coils we can gather into a vector $\mathbb{I}$ in a space we call the \emph{current space}. Similarly, the 27 readouts of the magnetic field can be gathered into a vector $\mathbb{B}$ in the \emph{field space}. Then, the matrix $M$ provides the transition from the current into the field space. In particular, we have:
\begin{equation}
  \mathbb{B} = M \mathbb{I} + \mathbb{B}_\text{offset} \ .
\end{equation}
Which is read: The measured values are a linear combination of the currents in the coils plus the ambient field.
The other direction, from the field to the current space, cannot be done exactly. Nevertheless, the optimal, in the least-squares sense, transition can be done with the use of the Moore--Penrose inverse (pseudoinverse) of the matrix $M$, denoted $M^\dagger$~\cite{penrose_1955}.
\marginpar{Finding the pseudoinverse of a matrix $A$ is equivalent to finding least-squares solution of a system of linear equations described by the matrix $A$ (although more computationally complex).}
In other words, the matrix $M$ tells us what field, as measured by the sensors, will a given set of currents produce. The matrix $M^\dagger$ tells us what currents to apply to best approximate a given field.

The use of the matrix $M$ was what distinguished the PSI's SFC system from other systems~\cite{Kelha1982,Brake1991,Spemann2003,Brys2005,Kobayashi2012,Voigt2013}, which used either one component of a sensor, or a weighted combination of several components, to control one coil.



\section{The feedback algorithm}
The PSI's SFC system followed the established norm~\cite{Kelha1982,Brake1991,Spemann2003,Brys2005} and used a PID loop to control the currents. (PI in this particular case, the derivative term was not used.) The $j$th current in the $n$th iteration was~\cite{Franke2013}:
\begin{equation}
  \label{eq:old_SFC_feedback}
  I^n_j = I^0_j +
    \underbrace{ \alpha^\mathcal{P}_j \, \Delta I^n_j }_\text{proportional term} +
    \underbrace{ \alpha^\mathcal{I}_j \, \sum_m \Delta I^m_j }_\text{integral term} \ .
\end{equation}

$\Delta I^n_j$ is the error value in the current space. It was obtained by using the error value in the field space (the difference of the measured and target fields) and transforming it into the current space with the pseudoinverse of the matrix~\cite{Franke2013}:
\begin{equation}
  \Delta I^n_j = \sum_{k'} \hat{M}^{-1}_{jk'} \, \Delta B^n_{k'} \ .
\end{equation}

\marginpar{The system when turned on was always static. Then the currents could be changed manually to achieve a desired field (or chosen from a predefined set). Then the system was switched to the dynamic mode.~\cite{Franke2013}}
Besides the proportional and integral terms the feedback formula also includes the $I^0_j$ term. It seems weird to give such a big of a role to particular currents that happened to be there at the zeroth iteration. It has to do with the following property of the system: the target field was always the one measured at the moment of switching the system from dynamic to static.
At that moment the error value was zero, and so was the integral term, so, according to the Eq.\,\ref{eq:old_SFC_feedback}, the output currents would immediately have switched to zero. In result, the system would violently destabilise. The additional term $I^0_j$ prevents that from happening.



\section{The spectrum of the SFC matrix}
\label{sec:nedm_sfc_matrix}
The feedback matrix used during the data taking of the nEDM-at-PSI experiment (2014--17) was the one measured by Franke~\cite{Franke2013}. This matrix not only needed to be inverted, but also additionally regularised. While it has been thoroughly discussed how to do the regularisation, the question of why was it needed at all was neither posed nor answered.

Let us elaborate on regularisation. The feedback matrix represents coefficients in a system of linear equations that need to be solved in order to make the optimal transition from the field to the current space. As the system of equations is over-determined, the best solution is found by the least-squares, which is equivalent to calculating the pseudoinverse matrix. A solution can be found reliably if the system is well-defined, i.e. the least-squares well is steep in every direction in the parameter space. If the well becomes a valley in some directions, the solution is not globally well-defined, although it still may be defied up to a parameter (the one pointing in the direction of the valley). We then speak of an ill-defined set of equations, or an ill-defined matrix. Regularisation helps an ill-defined problem to become better defined, at the cost of the solution having larger sum of the residuals squared.

A real matrix $M$ may be decomposed into $M = U S V^T$, where $U$ and $V$ are unitary, and $S$ is diagonal. This is called the \emph{Singular Value Decomposition} (SVD). The singular values lie on the diagonal of $S$, which is called the \emph{spectrum}. The spectrum of the feedback matrix is of uttermost importance. Pseudo-inverting a matrix with small singular values is an ill-posed problem. Similarily, as inverting small numbers.

One defines the \emph{condition number} of a matrix as the ratio of extreme values of its spectrum. For a matrix with a flat spectrum, all singular values equal, the condition number is 1 and the set of linear equations this matrix represents is well defined. The more they differ, the higher the condition number and the worse defined the problem is. The effect is that the noise in the original matrix becomes amplified by the condition number in the pseudoinverse.

Figure~\ref{fig:nEDM_SFC_svd} presents the spectrum of the PSI's SFC feedback matrix. First thing to note is that its condition number is $9.6 / 0.51 = 18.2$. A factor of 18 amplification of noise clearly explains why regularisation was necessary.

It is interesting to ask \emph{why} the system had an ill-defined feedback matrix. A very small singular value means that there exists a coil, or a combination thereof, which has about 18 times smaller influence on the field then others. In Fig.\,\ref{fig:nEDM_SFC_svd} the columns are the singular values with their corresponding coil-vectors. The first three, starting from the left, are easiest to interpret. Each of them is a pair of coils configured as Helmholtz-pair with roughly the same current, producing a homogeneous field in each of the spatial directions.
%The smaller singular value, or the effect on the field, in the Y direction is explained by the fact that this is the longitudinal axis of the μ-metal cylinder.
The last column has also a clear interpretation: it corresponds to all pairs configured as anti-Helmholtz, with currents flowing in the opposite directions. The magnetic field that it created was a complicated and high-order. The fact, that this combination has so little influence on the measured field means, that it hardly changes any solution for currents when added upon it. It spans a valley in the parameter space in the least-squares problem. If not regularised, a small change in the measured field would cause a large change in the currents along the valley. Rapid oscillations in this direction would render the system unstable.

\begin{figure}
  \centering
  \includegraphics[width=.7\linewidth]{gfx/nEDM_SFC/coil-singular_vectors_of_the_nEDM_SFC_matrix}
  \caption
  [TODO]
  {The coil-singular values of the SFC matrix. Columns correspond to singular combinations of the coils. For each column the corresponding singular value is indicated.}
  \label{fig:nEDM_SFC_svd}
\end{figure}

It is to note that the feedback matrix is defined solely by the configuration of the coils, sensors and ferromagnetic materials. It follows that care has to be taken already at the design stage to create a system that will be well defined.

\note{Propose here immediately a solution: orthogonal system for elements of expansion! Any combination of 1st order gradients is a 1st order gradient.}



\section{A good set of coils for an arbitrary field}
When one wants to be able to produce an arbitrary field, it is best to use an orthogonal expansion, in analogy to the Taylor expansion for functions. Furthermore, it is desired that the terms of the expansions are natural for the system considered (as spherical harmonics used to describe the wavefunctions of electrons in atomic systems are solutions to the Schrödinger equation themselves). Also, from the point of view of control, it is much easier to control an orthogonal system. See eg. \,\citep{Branch1984} for detailed motivation (really?). In an expansion one can gradually improve the system.

Such an expansion exists: \emph{carthesian harmonic polynomials}.
\begin{equation}
  \mathbf{B}(\mathbf{r}) = \sum_{n}\,H_n\mathbf{P}_n(\mathbf{r})
\end{equation}
where $H_n$ are the expansion coefficients. They satisfy Maxwell's equations, so it is always theoretically possible to create a coil that produces the eigenfunctions. The table~\ref{tab:coils_carthesian_harmonics} presents first few of them.

\begin{table}
  \centering
  \begin{tabular}{c|ccc}
    n & $P_n^x(\mathbf{r})$ & $P_n^y(\mathbf{r})$ & $P_n^y(\mathbf{r})$ \\ \hline
    1 & 1 & 0 & 0 \\
    2 & 0 & 1 & 0 \\
    3 & 0 & 0 & 1 \\ \hline
    4 & $x$ &  0  & $-z$ \\
    5 & $y$ & $x$ &   0  \\
    6 &  0  & $y$ & $-z$ \\
    7 & $z$ &  0  & $ x$ \\
    8 &  0  & $z$ & $-y$ \\
  \end{tabular}
  \caption{Carthesian harmonic polynomials.}
  \label{tab:coils_carthesian_harmonics}
\end{table}

The first three terms are homogeneous fields. Further five are the five independent gradients.




\section{Design challenges for n2EDM}
Give the situational plan.

Very tight space. Lots of spatial constraints. Very, very large!

This all calls for spetail coils.

An active magnetic field compensation system has a fixed set of coils. With them, it can only compensate these fields, which the coils can create. For example, it is natural to expect most of the variations in the field to come from sources that are far away, so that they are approximately homogeneous.
To compensate them, the coils need to create a homogeneous counteracting field.
However, for a pair of Helmholtz coils the size of the volume where the field is homogeneous on a 2\% level is only around 0.4 that of the size of the coils~\cite{Kirschvink1992}. In other words, an active compensation system can only stabilise the  field in a limited volume.

In nEDM-at-PSI the size of the coils was \SIrange[range-phrase = --]{6}{8}{\meter}, while the outermost $\upmu$-metal shield of the experiment was only around \SI{2}{\meter} large. In the n2EDM the shield is going to be \SI{5}{\meter} large, while the experimental area can hardly accommodate larger coils. So for the design of the n2EDM's active shielding it was necessary to come up with coils that would the field to be compensated in the whole volume occupied by the shield.