\chapter{Additional periodograms}
\label{ch:alp_appendix}

\begin{figure}[h!]
  \centering
  \includegraphics[width=\linewidth]{gfx/axions/AP_detection_and_no_GDC.png}
  \caption{Periodogram of the $R$ time series of the PSI experiment data, sensitive to oscillations in the quantity $d_\mathrm{n} - \left( \mu_\mathrm{n} / \mu_\mathrm{Hg} \right) \, d_\mathrm{Hg}$, taken with the $\mathbf{E}$ and $\mathbf{B}$ fields antiparallel (black line).
  The distribution of the MC-generated periodograms is depicted in green. The thick line depicts the mean, around which $1,2,…,5\,\sigma$ bands of the distribution are marked. around it marked. MC is used to calculate $1,2,…,5\,\sigma$ false-alarm thresholds, depicted in orange.
  % For clarity, we also plot the smoothed version in orange.
  % There are two regions where a rise in the amplitude is expected, namely around \SI{28}{\micro\hertz} (inverse of 10 hours) and \SI{3.3}{\milli\hertz} (inverse of 300 seconds), due to the time structure of the data taking (see the main text for more details).
  The periodogram of non-gradient-drift-corrected data is shown in pink. For details see Ch.\,\ref{ch:axion-analysis}.}\label{fig:app_AP_periodogram}
  % For clarity, we also plot the smoothed version in orange.}
\end{figure}

\begin{figure}[h!]
  \centering
  \includegraphics[width=\linewidth]{gfx/axions/E0_detection_and_no_GDC.png}
  \caption{Periodogram of the $R$ time series of the PSI experiment data, not sensitive to nEDM effects, taken with the $\mathbf{E} = 0$ (black line).
  The distribution of the MC-generated periodograms is depicted in green. The thick line depicts the mean, around which $1,2,…,5\,\sigma$ bands of the distribution are marked. around it marked. MC is used to calculate $1,2,…,5\,\sigma$ false-alarm thresholds, depicted in orange.
  % For clarity, we also plot the smoothed version in orange.
  % There are two regions where a rise in the amplitude is expected, namely around \SI{28}{\micro\hertz} (inverse of 10 hours) and \SI{3.3}{\milli\hertz} (inverse of 300 seconds), due to the time structure of the data taking (see the main text for more details).
  The periodogram of non-gradient-drift-corrected data is shown in pink. For details see Ch.\,\ref{ch:axion-analysis}.}\label{fig:app_E0_periodogram}
  % For clarity, we also plot the smoothed version in orange.}
\end{figure}

\begin{figure}[h!]
  \centering
  \includegraphics[width=0.9\linewidth]{gfx/axions/winddeltah4mm_Bup_detection.png}
  \includegraphics[width=0.9\linewidth]{gfx/axions/winddeltah4mm_Bdown_detection.png}
  \caption{The periodograms of the $R$ time series measured with the main magnetic field pointing upwards (top) and downwards (bottom). The distribution of the MC-generated periodograms is depicted in green. The thick line depicts the mean, around which $1,2,…,5\,\sigma$ bands of the distribution are marked. around it marked. MC is used to calculate $1,2,…,5\,\sigma$ false-alarm thresholds, depicted in orange. For details see Ch.\,\ref{ch:axion-analysis}.}\label{fig:app_wind_periodogram}
  % For clarity, we also plot the smoothed version in orange.}
\end{figure}

\begin{figure}[h!]
  \centering
  \includegraphics[width=0.9\linewidth]{gfx/axions/P_look-elsewhere.png}
  \caption{Quantitative illustration of the look-elsewhere effect in the $R$ time series of the PSI experiment data, sensitive to oscillations in the quantity $d_\mathrm{n} - \left( \mu_\mathrm{n} / \mu_\mathrm{Hg} \right) \, d_\mathrm{Hg}$, taken with the $\mathbf{E}$ and $\mathbf{B}$ fields parallel. See the caption of Fig.\,\ref{fig:P_look-elsewhere} for details.}\label{fig:app_P_look_elsewhere}
  % For clarity, we also plot the smoothed version in orange.}
\end{figure}

\begin{figure}[h!]
  \centering
  \includegraphics[width=0.9\linewidth]{gfx/axions/AP_look-elsewhere.png}
  \caption{Quantitative illustration of the look-elsewhere effect in the $R$ time series of the PSI experiment data, sensitive to oscillations in the quantity $d_\mathrm{n} - \left( \mu_\mathrm{n} / \mu_\mathrm{Hg} \right) \, d_\mathrm{Hg}$, taken with the $\mathbf{E}$ and $\mathbf{B}$ fields antiparallel. See the caption of Fig.\,\ref{fig:P_look-elsewhere} for details.}\label{fig:app_AP_look_elsewhere}
  % For clarity, we also plot the smoothed version in orange.}
\end{figure}

\begin{figure}[h!]
  \centering
  \includegraphics[width=0.9\linewidth]{gfx/axions/E0_look-elsewhere.png}
  \caption{Quantitative illustration of the look-elsewhere effect in the $R$ time series of the PSI experiment data, not sensitive to nEDM effects, taken with the $\mathbf{E} = 0$. See the caption of Fig.\,\ref{fig:P_look-elsewhere} for details.}\label{fig:app_E0_look_elsewhere}
  % For clarity, we also plot the smoothed version in orange.}
\end{figure}


% \section*{Further discussion}
% % \mnote{In this paragraph discuss the the distribution of the p-values and plot the filtered p-values.}
% % \mnote{Now, proceed to discuss the distribution of p-values. Then, proceed to plot the filtered p-values as the function of frequency.}
% There were a number of frequencies where statistically significant excesses were observed in the datasets sensitive to an oscillating nEDM\@.
% Even though the excesses are not compatible with an axion-induced signal, it is interesting to investigate the possible reason.

% In the first step we look at the p-value distribution, shown in black in Fig.\,\ref{fig:axions_P_p-values} (for the EB parallel dataset).
% The distribution is flat with narrow peak in the first bin, corresponding to the highest excesses.
% This means that there is only a small number (around 300 out of \num[detect-all=true]{156198}) frequencies, where more power than expected is observed.
% \marginpar{A global, like over- or underestimating the error-bars, would cause a tilt in the p-value distribution.}
% To identify this group we plot the ratio of the observed to average amplitude (Fig.\,\ref{fig:axions_observed-predicted_power_ratio_filtered}). The lines are low-pass filtered to average the statistical noise out. The excesses occur only for frequencies below \SI{1e-4}{\hertz} and in a narrow region around inverse \SI{300}{\second}. These are the regions where the gradient-drift correction had an effect.

% Here tell how it might be due to the imperfect gradient-drift correction. Say, that the narrow region around inverse \SI{300}{\second} is most likely strongly correlated with the low frequencies (essentially peak folding around the Nyquist frequency in the case of evenly sampled data.)


% % \begin{figure}
% %   \centering
% %   \includegraphics[width=0.9\linewidth]{gfx/axions/P_p-values.pdf}
% %   \caption{Accounting for the look-elsewhere effect for the parallel dataset. It provides the minimal local p-value -- global p-value transition. The fit with the model \dots gives the number of frequencies. \note{Prepare this histogram in a way, that the curves for all the datasets are shown. Then may need to normalise it.}}\label{fig:axions_P_p-values}
% % \end{figure}

% % \begin{figure}
% %   \centering
% %   \includegraphics[width=0.9\linewidth]{gfx/axions/observed-predicted_power_ratio_filtered.pdf}
% %   \caption{Maybe mark the 1/300 seconds. Note, that because of the settling of the filter first $N$ points need to be dropped, so not whole range is plotted.}
% %   \label{fig:axions_observed-predicted_power_ratio_filtered}
% % \end{figure}



% The idea is to give them all for completeness.
% \mnote{Also, put the axion wind plots here!}


% \section*{Axion-wind}
% \begin{figure}
%   \centering
%   \includegraphics[width=0.9\linewidth]{gfx/axions/winddeltah4mm_Bdown_detection.png}
%   \includegraphics[width=0.9\linewidth]{gfx/axions/winddeltah4mm_Bup_detection.png}
%   \caption{The periodogram of the $R$ time series measured with the main magnetic field pointing upwards. The distribution of MC-generated periodograms, assuming no signal, is depicted in green (the 1, 2 and 3$\upsigma$ bands and the mean with a line). The 1,2,…,5$\upsigma$ false-alarm thresholds are depicted in orange.}\label{fig:axions_wind_detection}
%   % For clarity, we also plot the smoothed version in orange.}
% \end{figure}



% \section*{EB antiparallel}
% \begin{figure}
%   \centering
%   \includegraphics[width=\linewidth]{gfx/axions/AP_detection_and_no_GDC.png}
%   \caption{\ldots}
%   \label{fig:axions_AP_detection_and_no_GDC}
% \end{figure}
% \begin{figure}
%   \centering
%   \includegraphics[width=\linewidth]{gfx/axions/AP_detection_area1.png}
%   \caption{\ldots}
%   \label{fig:axions_AP_detection_area1}
% \end{figure}
% \begin{figure}
%   \centering
%   \includegraphics[width=\linewidth]{gfx/axions/AP_detection_peak_62.png}
%   \caption{\ldots}
%   \label{fig:AP_detection_peak_62}
% \end{figure}
% \begin{figure}
%   \centering
%   \includegraphics[width=\linewidth]{gfx/axions/AP_detection_peak_144968.png}
%   \caption{\ldots}
%   \label{fig:AP_detection_peak_144968}
% \end{figure}


% \section*{EB parallel}

% \begin{figure}
%   \centering
%   \includegraphics[width=\linewidth]{gfx/axions/P_detection_and_no_GDC.png}
%   \caption{\ldots}
%   \label{fig:P_detection_and_no_GDC}
% \end{figure}
% \begin{figure}
%   \centering
%   \includegraphics[width=\linewidth]{gfx/axions/P_detection_peak_21.png}
%   \caption{\ldots}
%   \label{fig:P_detection_peak_21}
% \end{figure}
% \begin{figure}
%   \centering
%   \includegraphics[width=\linewidth]{gfx/axions/P_detection_peak_123.png}
%   \caption{\ldots}
%   \label{fig:P_detection_peak_123}
% \end{figure}
% \begin{figure}
%   \centering
%   \includegraphics[width=\linewidth]{gfx/axions/P_detection_peak_148.png}
%   \caption{\ldots}
%   \label{fig:P_detection_peak_148}
% \end{figure}
% \begin{figure}
%   \centering
%   \includegraphics[width=\linewidth]{gfx/axions/P_detection_peak_145389.png}
%   \caption{\ldots}
%   \label{fig:P_detection_peak_145389}
% \end{figure}
% \begin{figure}
%   \centering
%   \includegraphics[width=\linewidth]{gfx/axions/P_detection_peak_145432.png}
%   \caption{\ldots}
%   \label{fig:P_detection_peak_145432}
% \end{figure}
% \begin{figure}
%   \centering
%   \includegraphics[width=\linewidth]{gfx/axions/P_detection_peak_145440.png}
%   \caption{\ldots}
%   \label{fig:P_detection_peak_145440}
% \end{figure}


% \section*{E=0}

% \begin{figure}
%   \centering
%   \includegraphics[width=\linewidth]{gfx/axions/E0_detection_and_no_GDC.png}
%   \caption{\ldots}
%   \label{fig:E0_detection_and_no_GDC}
% \end{figure}
% \begin{figure}
%   \centering
%   \includegraphics[width=\linewidth]{gfx/axions/E0_look-elsewhere.png}
%   \caption{\ldots}
%   \label{fig:E0_look-elsewhere}
% \end{figure}
% \begin{figure}
%   \centering
%   \includegraphics[width=\linewidth]{gfx/axions/E0_best_signal_candidate.png}
%   \caption{\ldots}
%   \label{fig:E0_best_signal_candidate}
% \end{figure}



% \chapter{Comparison of the analyses}


% \chapter{Sidereal frequency analysis}
% \begin{figure}
%   \centering
%   \includegraphics[width=0.8\linewidth]{gfx/axions/winddeltah4mm_sidereal.pdf}
%   \caption{\cite{Romalis2009_NF}}\label{fig:axions_sidereal_detection}
% \end{figure}

% \begin{figure}
%   \centering
%   \includegraphics[width=0.8\linewidth]{gfx/axions/winddeltah4mm_sidereal_detection.pdf}
%   \caption{\cite{Romalis2009_NF}}\label{fig:axions_sidereal_cdf}
% \end{figure}

% A note about the signals seen at this particular frequency. Refer to the Lorentz-invariance paper, improve the limit directly?~\cite{ALTAREV20112365}

% Another magnetic-field-like effect\ldots

% At the moment (21 Feb 2018) it seems, that there is a signal exactly at the sidereal frequency. The periodograms are in Fig.\,\ref{fig:axions_sidereal_detection}. It is around 4$\upsigma$ in both datasets, the CDF of the B up dataset is in Fig.\,\ref{fig:axions_sidereal_cdf}. The phase relation is correct (around \ang{160}, but I don't know the uncertainty on that number). In the B up dataset it sticks prominently out exactly there. In the B down dataset it is a bit problematic, as this happens to be right at the edge of some wider excess in power, so this peak is surrounded by a couple more, some even more significant. I could cut the data to get this peak out. 