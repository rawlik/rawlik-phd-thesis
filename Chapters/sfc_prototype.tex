% !TEX root = ../rawlik-phd-thesis.tex
\chapter{Next generation active magnetic field compensation}
\label{ch:sfc-prototype}

The coil design method described in the previous chapter opened the door to a next generation of active magnetic field compensation systems. Ones where the coil system is not much larger than the fiducial volume and where high-order terms of the magnetic field can be compensated, all while retaining a low number of controlled degrees of freedom.

The large fiducial volume is of particular importance for the n2EDM experiment at PSI. The available area\ldots

In the laboratory at ETH Zürich an active magnetic field compensation system was constructed, based on the coil design method described earlier. The goal was do demonstrate the feasibility of construction of the coils, develop and test the electronic and software parts of the feedback algorithm, and test the whole system.



\section{The first iteration -- coil structure}
When discussing the the coil design method we have already indicated a possible way to realise it in practice---construct a grid out of cable channels. The system, pictured in Fig.\,\ref{fig:prototype_photo}, was built as a $5 \times 9 \times 5$ square tiles. The tile size\ldots The total size\ldots
\mnote{Introduce the coordinate system somewhere here.}

\begin{figure}
  \centering
  \includegraphics[width=0.9\linewidth]{gfx/prototype/DSC03472.JPG}
  \caption{The active magnetic compensation system in the laboratory at ETH Zürich. It consists cable channels mounted on an aluminum support structure. In the channels copper wires making up the coils can be seen. On the left-hand side the control cabinet of the system is visible. Indicate the coordinate system here!}
  \label{fig:prototype_photo}
\end{figure}

The support frame was made of aluminum construction profiles. Thereon, on each side, was a large one-piece aluminum sheet, with square cut-outs leaving the material only directly below the cable channels. \note{Explain better, tricky.}

\begin{figure}
  \centering
  \includegraphics[width=0.9\linewidth]{gfx/prototype/coil_y_currents.png}
  \caption{The optimal current net for the $y$-coil of the ETH active magnetic field compensation system. Both $5 \times 5$ faces ($y = \mathrm{const}$ planes) are identical and are depicted on the left-hand side. The rectangular faces are identical, too, and are depicted on the right-hand side. For each segment the current per \SI{100}{\micro\tesla} of generated field is indicated upwards and to the right. CHECK}
  \label{fig:prototype_coil_y_currents}
\end{figure}

In its first version the system featured three coils for the homogeneous components of the magnetic field. The coils were designed following the method described in Ch.\,\ref{ch:coil_design} (excluding the simplification algorithm, as explained later). The fiducial volume was chosen to be a cuboid, centred in the system, \ldots away from the walls. The optimal current net for the $y$-coil is depicted in Fig.\,\ref{fig:prototype_coil_y_currents}. At the time when the system was constructed the simplification algorithm, as described in Ch.\,\ref{ch:coil_design} had not been developed, yet. The decomposition, shown in Fig.\,\ref{fig:prototype_coil_y_decomposition} was done arbitrarily. From the symmetry the $x$ and $z$ coils are identical. The current net are depicted in Fig.\,\ref{fig:prototype_coil_x_z_currents} and the decomposition in Fig.\,\ref{fig:prototype_coil_x_z_decomposition}.

\begin{figure}
  \centering
  \includegraphics[width=0.9\linewidth]{gfx/prototype/coil_y_decomposition.png}
  \caption{The decomposition of the current net of the $y$-coil (Fig.\,\ref{fig:prototype_coil_y_currents}) into current loops. For clarity only a sixteenth part of the system is depicted, all others being identical on the grounds of symmetry. Each loop is indicated with a different colour. The currents are given per \SI{50}{\micro\tesla} of generated field. SHOW WHICH PART IS DEPICTED ON A SMALL 3D DRAWING}
  \label{fig:prototype_coil_y_decomposition}
\end{figure}

\begin{figure}
  \centering
  \includegraphics[width=0.9\linewidth]{gfx/prototype/coil_x_z_currents.png}
  \caption{Mention, that it shows\ldots}
  \label{fig:prototype_coil_x_z_currents}
\end{figure}

% up to the point of the simplification algorithm. Here, The decomposition into loops was done by exploiting symmetries of the system. It is suboptimal in the sense, that there are more windings than there could be.

% Then describe how the coils were designed. The optimized current net is shown in Fig\,\ref{fig:prototype_coil_y_currents} (for the y-coil) and \ref{fig:prototype_coil_x_z_currents} (the x- and z-coils).

% It follows the already described coil design method, up to the point of the simplification algorithm. Here, The decomposition into loops was done by exploiting symmetries of the system. It is suboptimal in the sense, that there are more windings than there could be. The decompositions are depicted in Figs.\,\ref{fig:prototype_coil_y_decomposition} (the y-coil) and \ref{fig:prototype_coil_x_z_decomposition} (the x- and z-coils).

\begin{figure}
  \centering
  \includegraphics[width=0.9\linewidth]{gfx/prototype/coil_x_z_decomposition.png}
  \caption{Mention, that it shows\ldots}
  \label{fig:prototype_coil_x_z_decomposition}
\end{figure}

Finally, the individual loops were discretised into \SI{5}{\ampere}, \SI{1}{\ampere} and \SI{0.1}{\ampere}, all per nominal \SI{50}{\micro\tesla}. For example, the \SI{19.48}{\ampere} current, indicated in black in Fig.\,\ref{fig:prototype_coil_y_decomposition}, was realised as three windings of the \SI{5}{\ampere} wire, four of the \SI{1}{\ampere} one and five of the \SI{0.1}{\ampere} one.

An enameled was then laid in the cable channels according to the discretised design. For one current component of a coil, e.g.\ \SI{5}{\ampere} in the $x$-coil, a single, long piece of wire was laid, making up all the windings of all loops. For the three coils, each with three components, nine longs wires were laid, in total.
\marginpar{Wire thickness was\ldots}

Here would be a place for the close-up! and a close-up in \note{put a photo of a close-up, too}.


\section{Mapping}
To verify that the coils indeed produce a field of the required homogeneity they were mapped. For that purpose a robot has been built---a fluxgate on an xyz-table, controlled with stepper motors. The robot, called \emph{mapper}, is pictured in Fig.\,\ref{fig:prototype_photo_inside}.

\begin{figure}
  \centering
  \includegraphics[width=0.9\linewidth]{gfx/prototype/DSC03476.JPG}
  \caption{The inside of the active magnetic field compensation system. In the centre the mapping robot (\emph{mapper}) is visible, with a fluxgate magnetic field sensor mounted to a movable platform. Mounted on black, vertical beams there are eight fluxgates for the active feedback.}
  \label{fig:prototype_photo_inside}
\end{figure}

A beam seen in the middle of Fig.\,\ref{fig:prototype_photo_inside} could move along the $y$ direction by the means ofith two timing belts fixed to it. The belts were wound around pulleys with stepper motors, visible to the left. Along the beam, the $x$ direction, a cart was moved in the same way---a timing belt and a stepper motor. On the cart three vertical rods were attached, with a plastic (POM) platform tightly threaded around them. On the platform there was a fluxgate magnetic field sensor. In the middle of the rods there was an aluminum threaded rod, passing through a threaded hole in the platform and mounted to a stepper motor on the cart. As the motor spun the threaded rod, the platform moved vertically, along the $z$ direction.

A simple kind of map possible with the mapper is a linear one, whereby the fluxgate was moved along one direction only. A map of the $y$-coil, mapped along the $y$ direction (in the middle of $x$ and $z$), is shown on the right-hand side in Fig.\,\ref{fig:prototype_linear_map} (only the $y$-component of the magnetic field is plotted, the coil was set to produce \SI{50}{\micro\tesla}). The background field was also mapped and the plotted map has it already subtracted. The field stayed in a \SI{\pm 0.2}{\micro\tesla} range around the average value.

\begin{figure}
  \centering
  \includegraphics[width=0.9\linewidth]{gfx/prototype/linear_map.png}
  \caption{Right-hand side: Linear map of the homogeneous field $y$-coil. The map is along the $y$-direction, in the middle of $x$ and $z$. The $y$-component of the magnetic field is plotted. On the left-hand side additionally maps with only the \SI{5}{\ampere} component coil, and with \SI{5}{\ampere} and \SI{1}{\ampere} components coil are shown.}
  \label{fig:prototype_linear_map}
\end{figure}

On the left-hand of Fig.\,\ref{fig:prototype_linear_map} additionally the maps with only the \SI{5}{\ampere} component coil, and with \SI{5}{\ampere} and \SI{1}{\ampere} components coil are shown. It is interesting to note how much of the field is produced by each of the components. In the middle region the \SI{5}{\ampere}, \SI{1}{\ampere} and \SI{0.1}{\ampere} components produced 86\%, 11\% and 3\% of the field, respectively. At the edge the shares change to 70\%, 16\% and 14\%. \note{These numbers give also the relative requirements for the components. Five times less stable? Makes no sense\ldots What would the requirements really be? Linearity, for example! Could use worse power supplies.}

Another type of map is a planar one. A horizontal map, in the middle $xy$-plane, of the $y$-coil is presented in Fig.\,\ref{fig:prototype_plane_map}. The plot shows the maximum deviation among all three components of the magnetic field, i.e.\ in the area enclosed by the \SI{1}{\micro\tesla}~isocountour all components of the field are within \SI{1}{\micro\tesla} from the nominal field. The border of the plot corresponds to planes where the wires of the coils are. The fiducial volume is marked with grey lines. The field inside the fiducial volume stays within \SI{1}{\micro\tesla} of the nominal one of \SI{50}{\micro\tesla}, so the relative homogeneity is 2\%.
\mnote{Need to be clear about the specification---it wasn't supposed to be 2\% EVERYWHERE! Maybe bring forward the specification plot that decided on the grid size?}

\begin{figure}
  \centering
  \includegraphics[width=0.9\linewidth]{gfx/prototype/plane_map.png}
  \caption{A horizontal map, in the middle $xy$-plane, of the $y$-coil. The maximum deviation among all three components of the magnetic field is plotted, i.e.\ in the area enclosed by the \SI{1}{\micro\tesla}~isocountour all components of the field are within \SI{1}{\micro\tesla} from the nominal field. The border of the plot corresponds to planes where the wires of the coils are. The fiducial volume is marked with grey lines.}
  \label{fig:prototype_plane_map}
\end{figure}



\section{DAQ}
Before proceeding to discuss the active stabilisation system, we devote a section to the hardware. These are technical details, but for the more tech-savvy of the readers haveing the hardware setting given first may ease following the discussion.

Already in Sec. \note{ref to the beginning} we set the frame of the active magnetic field stabilisation system. A change in the magnetic field is detected with an array of sensors, an appropriate response is calculated and applied by driving a change in currents flowing in the coils. The feedback loop is closed through the air when the sensors detect the change in the field caused by the coils.

\marginpar{In a fluxgate a ferromagnetic core is periodically driven into saturation. When it is not saturated, it is highly permeable and sucks the external magnetic flux in. When saturated, it does not occur. A pickup coil detects the changes in the external flux as it is alternately sucked in and out of the core.}

The field was measured with eight fluxgates, visible in Fig.\,\ref{fig:prototype_photo_inside}.
The sensors were Stefan Mayer Instruments FLC3-70 three-axis fluxgates, \SI{\pm 200}{\micro\tesla} range, \SI{1}{\kilo\hertz} bandwidth, $\pm 1\% \pm \SI{0.5}{\micro\tesla}$.
They were powered with an in-house--built double \SI{\pm 5}{\volt} power supply. The outputs of the fluxgates were \SI{\pm 5}{\volt} signals proportional to the magnetic field.
The analogue signals were directly digitised with 16-bit Beckhoff EL3602 24-bit differential analogue-to-digital converters (ADCs). The digital information was collected in software running on a PC computer.

The software stack, running under OpenSUSE Linux, consisted of a low-level Ethercat driver~\cite{etherlabcode} on top of which a custom program, written in \texttt{julia}~\cite{julia}, was running. This setup was optimised for high flexibility and close-to-zero turnaround time. In particular, it was possible to develop the software interactively \emph{while} the system was running. Besides the usual recording and graphically plotting the data, the main task of the software was to evaluate the optimal response for the measured magnetic field changes. The optimal response, the new currents to be applied to the coils, was sent to the digital-to-analogue converters (DACs).

\marginpar{The main julia program published the data on a \texttt{ZMQ} \texttt{PUB} socket. The data were stored with a separate \texttt{julia} program, collecting the data on a \texttt{ZMQ} \texttt{SUB} socket. Similarily, the plotting program was separate, written in \texttt{Python}.}

The DACs were 16-bit Beckhoff EL4134. The signals were then amplified with an array of four-quadrant SERVOWATT amplifiers, configured to amplify the voltage input to a current output. \note{Put here the exact amplifiers}. The currents were then directly fed into the coils.

\begin{SCfigure}
  \centering
  \includegraphics[width=0.4\linewidth]{gfx/prototype/DSC03477_cropped.jpeg}
  \caption{The 19-inch cabinet hosting the electronics for the active magnetic field stabilisation system. From top: a \SI{24}{V} power supply for the Beckhoff EtherCAT clamps and in-house--built \SI{\pm 5}{\volt} power supply for the fluxgates; spare EtherCAT clamps and control system for the mapper; a large array of EtherCAT DACs and ADCs for the active stabilisation system; breakout panel for the fluxgate cables (RJ-45 plugs); subrack with \SI{\pm 0.2}{A} amplifiers; subrack with \SI{\pm 2}{A} amplifiers; two subracks with \SI{\pm 10}{A} amplifiers; a general-purpose Kepco \SI{\pm 20}{V} \SI{\pm 20}{A} four-quadrant amplifier; the PC.}
  \label{fig:prototype_photo_daq}
\end{SCfigure}

The hardware was hosted in a 19-inch cabinet, pictured in Fig.\,\ref{fig:prototype_photo_daq}. A detailed list of the components can be found in the Figure's caption. This is all that, additionally to the coils themselves, was used to set up the active magnetic field compensation.


\section{The SFC matrix}
The central element of the active magnetic field compensation is the response matrix of the system, also called the SFC matrix \note{need to establish the nomenclature}. It is the matrix of proportionality constants between the current in each coil and the readout of each magnetic field sensor. The SFC matrix of the nEDM@PSI SFC system and its properties were thoroughly discussed in Sec.\,\ref{fig:nEDM_SFC_matrix}.

There is a fundamental difference between the matrix of the nEDM@PSI system and the next generation one. The nEDM@PSI's matrix was not known a priori. The system was constructed and the matrix was measured as the system's property. This could be considered as the reason that the matrix was ill-conditioned and the system required regularisation and PI control to be stable. Conversely, the new system takes the matrix into account already at the design phase. The coils are designed to produce magnetic field corresponding to the terms of the Cartesian harmonic expansion of the field. \note{Needs to be clear here!} This ensures that they are orthogonal to one another. We expect the matrix to have the condition number equal to 1.

\note{Put a figure about the measurement of the matrix?}

The active magnetic field compensation system constructed at ETH implemented a new way of measuring the matrix. Not only it changes currents in all the coils simultaneously, shortening the duration of the procedure, but can also measure in close-to-zero field conditions.

To measure the matrix we consider the space spanned by all the coils. In this case it is three-dimensional---the current in the $x$-coil, in the $y$-coil and in the $z$ one. A point in this space corresponds to one configuration of the currents in the coils. We pick a set of points on an n-sphere in this space and continuously change the currents to walk from one point to another. As all the currents are changed simultaneously, the measurements of the sensors are recorded. Then for each readout channel ($8 \times 3 = 24$ in total) a linear model in fitted to estimate the proportionality constants between the readout and the currents in the coils:
\begin{equation}
  B_i = B_i^0 + \sum_{j=x,y,z} \mathbb{M}_{i,j} \, I_j \ ,
\end{equation}
where $B_i$ is the readout channel ($i$ going from 1 to 24), $I_j$ are the currents in the coils ($j$ going over $x$, $y$ and $z$) and $\mathbb{M}_{i,j}$ is the matrix of proportionality constants---the SFC matrix. $B_i^0$ is the free offset vector---the background field.
\note{Sort out the indexes, which is which.}

If the system is fully linear it doesn't matter. The center point and the radius. Typically the procedure: measure for (how long? 5-10 seconds) around the zero-currents point. Then measure around the zero-field points, by inverting the matrix! Here give the equation! Finally, reduce the radius of measurement. The nonlinearities may matter in a presence of mu-metal.

A typical measured matrix is\ldots Estimate the field inhomogeneity right away from it! Give two at two distances, maybe?



Here describe the procedure of measuring the matrix, as well as the matrix itself.

Describe how the matrix is inverted to measure near the zero-field! Describe traversing the sphere and how it get smaller and smaller.

Then continue to analyse the matrix. Discuss the SVD decomposition and its singular values. Or is it the right place to do it here?


\section{The feedback algorithm}
Say, that implemented in software (julia). Write down the equation.

Write here about the measured delay and that it was crucial for the stability!


\section{Dynamic stabilisation}

Then show the dynamic stabilisation plot. A large permanent magnetic dipole was constructed (two neodymium magnets connected with an iron rod). It was moved

\begin{figure}
  \centering
  \subfloat{
    \label{fig:prototype_compensation_time_1}
    \includegraphics[width=.45\linewidth]{gfx/prototype/uncompensated_7_5m.png}}
  \quad
  \subfloat{
    \label{fig:prototype_compensation_time_2}
    \includegraphics[width=.45\linewidth]{gfx/prototype/compensated_7_5m.png}}
  \caption{Following the algorithm to simplify a coil. The left column shows the net of a current with the total current along edges of tiles. In each iteration the loop with the highest current is found and transferred onto the simplified solution, shown in the right column. We show iterations, from top: zeroth, fourth and eighth.}
  \label{fig:prototype_compensation_time}
\end{figure}


\begin{figure}
  \centering
  \subfloat{
    \label{fig:prototype_compensation_performance}
    \includegraphics[width=.45\linewidth]{gfx/prototype/big_magnet_performance.pdf}}
  \quad
  \subfloat{
    \label{fig:prototype_shielding_factor}
    \includegraphics[width=.45\linewidth]{gfx/prototype/big_magnet_shielding_factor.pdf}}
  \caption{Following the algorithm to simplify a coil. The left column shows the net of a current with the total current along edges of tiles. In each iteration the loop with the highest current is found and transferred onto the simplified solution, shown in the right column. We show iterations, from top: zeroth, fourth and eighth.}
  \label{fig:prototype_compensation}
\end{figure}


\section{Long-term Stability}
Write about the stability. Put the stability plot. State, that it does not require the output to be stable -- the stability is defined by the stability of the input. Say, that the level of stability reached with the system corresponds to the fluxgates' specified at 0.1K temperature drift - cannot be expected to be better than that in the laboratory.

\begin{figure}
  \centering
  \includegraphics[width=0.9\linewidth]{gfx/prototype/run7_field_stability.pdf}
  \caption{Mention, that it shows\ldots}
  \label{fig:prototype_stability}
\end{figure}



\section{Open-design cage}

