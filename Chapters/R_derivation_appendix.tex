\chapter{R derivation}
\label{ch:R_derivation_appendix}

Here we derive the expression for the ratio of the neutron and mercury precession frequencies
\begin{equation}
  R = \frac{\nu_\text{n}}{\nu_\text{Hg}} \ .
  \label{eq:appendix_R_definition}
\end{equation}

For a spin $S = \tfrac{1}{2}$ particle in a combination of an electric and magnetic field we have:
\begin{equation}
  H = - 2 \left( \mu \, \mathbf{B} + d \, \mathbf{E} \right ) \cdot \mathbf{S} \ .
\end{equation}
For the parallel (p) and antiparallel (ap) combination of the fields:
\begin{equation}
  H_\text{p, ap} = - \mu B \pm d E \ .
\end{equation}
The transition between the spin-up and spin-down states is twice the energy:
\begin{equation}
  H_{\text{p} \leftrightarrow \text{ap} } = h \nu = 2 \left( \mu B \pm d E \right) \ .
\end{equation}
Substituting into Eq.\,\ref{eq:appendix_R_definition}:
\begin{align}
  R &= \frac{\nu_\text{n}}{\nu_\text{Hg}} = \frac{ \frac{2}{h} \left( \mu_\text{n} B + d_\text{n} E \right) }{ \frac{2}{h} \left( \mu_\text{Hg} B + d_\text{Hg} E \right) } = \nonumber \\
      &= \frac{ \left( \mu_\text{n} B \pm d_\text{n} E \right) }{ \left( \mu_\text{Hg} B \pm d_\text{Hg} E \right) } = \nonumber \\
      &= \frac{ \mu_\text{n} B }{ \left( \mu_\text{Hg} B \pm d_\text{Hg} E \right) } \pm \frac{ d_\text{n} E }{ \left( \mu_\text{Hg} B \pm d_\text{Hg} E \right) } = \nonumber \\
      &= \frac{\mu_\text{n}}{\mu_\text{Hg}} \times \frac{1}{ 1 \pm \frac{d_\text{Hg} E}{\mu_\text{Hg} B} } \pm \frac{d_\text{n} E}{\mu_\text{Hg} B} \times \frac{1}{1 \pm \frac{ d_\text{Hg} E }{ \mu_\text{Hg} B}} = \nonumber \\
      &= \left\{ \frac{1}{1 \pm x} = 1 \mp x \right\} = \nonumber \\
      &= \frac{\mu_\text{n}}{\mu_\text{Hg}} \mp d_\text{Hg} \frac{\mu_\text{n}}{\mu_\text{Hg}} \frac{E}{\mu_\text{Hg} B} \pm d_\text{n} \frac{E}{\mu_\text{Hg} B} \mp d_\text{n} d_\text{Hg} {\left( \frac{E}{\mu_\text{Hg} B} \right)}^2 = \\
      &\approx \frac{\mu_\text{n}}{\mu_\text{Hg}} \pm \left( d_\text{n} \mp  d_\text{Hg} \frac{\mu_\text{n}}{\mu_\text{Hg}} \right) \frac{E}{\mu_\text{Hg} B} \ .
\end{align}
We approximate the magnetic field with
\begin{equation}
   h \nu_\text{Hg} \approx 2 \mu_\text{Hg} B \ \Rightarrow \ \mu_\text{Hg} B \approx \frac{h \nu_\text{Hg}}{2}
\end{equation}
and substitute to get
\begin{equation}
  R = \frac{\nu_\text{n}}{\nu_\text{Hg}} = \frac{\mu_\text{n}}{\mu_\text{Hg}} \pm \left( d_\text{n} \mp \frac{\mu_\text{n}}{\mu_\text{Hg}} \, d_\text{Hg} \right) \frac{2 E}{ h  \nu_\text{Hg}} + \Delta \ .
\end{equation}
