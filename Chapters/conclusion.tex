\addtocontents{toc}{\protect\vspace{\beforebibskip}} % Place the bibliography slightly 
% below the rest of the document content in the table of contents
% Add manually to toc, so that it does not get a letter as it were a part of an appendix
\addcontentsline{toc}{chapter}{\tocEntry{Conclusion}}

\begingroup

\let\clearpage\relax
\let\cleardoublepage\relax
\let\cleardoublepage\relax

\chapter*{Conclusion}
At the time of writing this thesis the analysis of the measurement of the electric dipole moment of the neutron performed at PSI was still ongoing. Once finished, it will give the most precise estimate of the nEDM to date. It might even be the first to hint a non-zero value. When compared with the values predicted by the various extensions of the Standard Model it will support some and contradict others, shedding light onto our understanding of the Universe. This will be an addition to a contribution already made as part of this work---the search for axion dark matter. At PSI the quest for the neutron electric dipole moment will continue with a new, more sensitive apparatus. A part of it is an active magnetic shield, whose design this work was also concerned with.

The new coil design method made it possible to incorporate active magnetic shields in tight spatial constraints. The grid-based design was practically demonstrated in a form of a small-scale active shield. This technology served as a base for a design of a shield for the n2EDM experiment at PSI\@. The n2EDM shield will perform better if it is tailored for the particular magnetic environment by featuring coils for high-order variations. To that purpose a magnetic field mapper was built---a mobile tower with sensors attached to it, which was used to map the field in the n2EDM areal.

The search for axion dark matter with the data measured in the nEDM experiment at PSI was the first of its kind. The ratio of the precession frequencies of stored neutrons and mercury atoms was checked for statistically significant, axion-induced oscillations. None were found, which resulted in the first laboratory limits for the axion coupling to gluons and an improvement on the ones on the axion-nucleon coupling. In the future a proposed resonant detection scheme may be used to use ultracold neutrons to search for heavier axions.

\endgroup