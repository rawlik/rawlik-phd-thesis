\chapter{Introduction}
\label{ch:axions-intro}

\marginpar{This section is largely based on and also reproduces text of Ref.\,\cite{PhysRevX.7.041034}.}
Based on astrophysical and cosmological observations an estimated \SI{26}{\percent} of the total energy density of the Universe and \SI{84}{\percent} of its mass content is dark matter (DM)~\cite{Planck2015}.
Observations give hints about the amount and distribution of DM, for example via rotational curves of galaxies or gravitational lensing~\cite{ApJ1990}, but the micro-scale properties of DM, in particular its constituents, remain unknown.

Among the candidates for DM is an axion, a new pseudoscalar particle, initially proposed to solve the strong QCD problem (the strong sector in the Standard Model appears to be fine-tuned to be $CP$-even)~\cite{PhysRevLett.38.1440,PQ1977B,Weinberg1978,Wilczek1978,Kim1979,Zakharov1980,Zhitnitsky1980B,Srednicki1981}. 
It has been later generalised to axion-like particles, or simply axions~\cite{Witten1984,Conlon2006,Witten2006,Arvanitaki2010,Arias2012,Marsh2015Review}.
Light ($m_a \lesssim \SI[per-mode=symbol]{0.1}{\electronvolt\per\clight\squared}$) axions can be produced efficiently via non-thermal production mechanisms, such as vacuum misalignment in the early Universe~\cite{Preskill1983cosmo,Sikivie1983cosmo,Dine1983cosmo}.
They fill the universe as almost stationary particles and through gravity participate in the galaxy formation. During the formation they gain speed ($\approx \SI[per-mode=symbol]{300}{\kilo\metre\per\second}$) and, as bosons, condensate into a coherent oscillating field ($\Delta\omega / \omega \sim \num{e-6}$)~\cite{Marsh2015Review}, with the frequency of the oscillation set by the mass of the axion $m_a$:
\begin{equation}
  a = a_0 \, \cos\left(\frac{ m_a c^2 }{\hbar} \ t\right) \ .
\end{equation}

Due to its effects on structure formation~\cite{Khlopov1985}, ultra-low-mass axion DM in the mass range $\SI{e-24}{\electronvolt} \lesssim m_a \lesssim \SI{e-20}{eV}$ has been proposed as a DM candidate that is observationally distinct from, and possibly favourable to, archetypal cold DM~\cite{Hu2000,Marsh2014,Schive2014,Marsh2015Review,Hui2017}.
The requirement that the axion de Broglie wavelength does not exceed the DM size of the smallest dwarf galaxies and consistency with observed structure formation~\cite{Marsh2015B,Schive2015,Marsh2017} give the lower axion mass bound $m_a \gtrsim \SI{e-22}{eV}$, if axions comprise all of the DM\@. However, axions with smaller masses can still constitute a sub-dominant fraction of DM~\cite{Hlozek15}.

It is reasonable to expect that axions interact non-gravitationally with standard-model particles.
Direct searches for axions have thus far focused mainly on their coupling to the photon (see the review~\cite{Axion-Review2015} and references therein).
\marginpar{Axions are hoped to convert into photons in a strong magnetic field. Helioscopes look for energetic axions produced in the sun. Haloscopes are sensitive to a relict axion dark matter.}
Recently, however, it has been proposed to search for the interactions of the coherently oscillating axion DM field with gluons and fermions, which can induce oscillating electric dipole moments (EDMs) of nucleons~\cite{Graham2011} and atoms~\cite{Stadnik2014A,Roberts2014A,Roberts2014B}, and anomalous spin-precession effects~\cite{Flambaum2013Patras,Stadnik2014A,Graham2013}.
The frequency of these oscillating effects is dictated by the axion mass, and more importantly, these effects scale linearly in a small interaction constant~\cite{Graham2011,Stadnik2014A,Roberts2014A,Roberts2014B,Flambaum2013Patras,Graham2013}, whereas in previous axion searches, the sought effects scaled quadratically or quartically in the interaction constant~\cite{Axion-Review2015}.

In this part two axion couplings are considered: the one to gluons and the one to nucleons:
\begin{align}
\label{Axion_couplings}
\mathcal{L}_{\textrm{int}} = \frac{C_G}{f_a} \frac{g^2}{32\pi^2} a G^{b}_{\mu \nu} \tilde{G}^{b \mu \nu}  - \frac{C_N}{2f_a} \partial_\mu a ~ \bar{N} \gamma^\mu \gamma^5 N \, ,
\end{align}
where $G$ and $\tilde{G}$ are the gluonic field tensor and its dual, $b=1,2,\ldots,8$ is the  color index, $g^2 / 4 \pi$ is the color coupling constant, {\color{black}$N$ and $\bar{N} = N^\dagger \gamma^0$ are the nucleon field and its Dirac adjoint,} $f_a$ is the axion decay constant, and $C_G$ and {\color{black}$C_N$} are model-dependent dimensionless parameters.
Astrophysical constraints on the axion-gluon coupling come from Big Bang nucleosynthesis~\cite{Blum2014,StadnikThesis,Stadnik2015D}:~$m_a^{1/4} f_a / C_G \gtrsim 10^{10}~\textrm{GeV}^{5/4}$ for $m_a \ll \SI{e-16}{eV}$ and $m_a f_a / C_G \gtrsim \SI{e-9}{GeV^2}$ for $m_a \gg \SI{e-16}{eV}$, assuming that axions saturate the present-day DM energy density,
and from supernova energy-loss bounds~\cite{Graham2013,Raffelt1990Review}:~$f_a / C_G \gtrsim \SI{e6}{GeV}$ for $m_a \lesssim 3 \times \SI{e7}{eV}$.
{\color{black}Astrophysical constraints on the axion-nucleon coupling come from supernova energy-loss bounds~\cite{Raffelt1990Review,Raffelt2008LNP}:~$f_a / C_N \gtrsim \SI{e9}{GeV}$ for $m_a \lesssim 3 \times \SI{e7}{eV}$, while existing laboratory constraints come from magnetometry searches for new spin-dependent forces mediated by axion exchange~\cite{Romalis2009_NF}:~$f_a / C_N \gtrsim 1 \times \SI{e4}{GeV}$ for $m_a \lesssim \SI{e-7}{eV}$. }

The axion-gluon coupling in Eq.\,\ref{Axion_couplings} induces the following oscillating EDM of the neutron via a chirally-enhanced 1-loop process
\footnote{Interaction in Eq.\,\ref{Axion_couplings} also non-perturbatively induces a mass $m_a \approx 6 C_G\,\si{\micro\electronvolt} \cdot (\SI{e12}{\giga\electronvolt} / f_a)$.
Axions with masses much smaller than this are theoretically fine-tuned.}~\cite{Witten1979,Witten1979B,Pospelov1999}:
\begin{equation}
\label{eq:nEDM_axion}
d_\mathrm{n}(t) \approx +2.4 \times 10^{-16} ~ \frac{C_G a_0}{f_a} \cos(m_a t) ~ \si{\elementarycharge\centi\metre} \, .
\end{equation}
The axion-gluon coupling also induces oscillating EDMs of atoms via the 1-loop-level oscillating nucleon EDMs and tree-level oscillating P,~T-violating intra-nuclear forces (which give the dominant contribution)~\cite{Stadnik2014A,Flambaum1984EDM,Flambaum1984EDMB}.
In the case of $^{199}$Hg, the oscillating atomic EDM is~\cite{Stadnik2014A,StadnikThesis,Flambaum1985EDM,Flambaum1985EDMB,Flambaum2002EDM,Dmitriev2003A,Dmitriev2003B,Dmitriev2005,Engel2005,Engel2010}
\begin{equation}
\label{199Hg-EDM_axion}
d_{\textrm{Hg}}(t) \approx +1.3 \times 10^{-19} ~ \frac{C_G a_0}{f_a} \cos(m_a t) ~ \si{\elementarycharge\centi\metre} \, ,
\end{equation}
which is suppressed compared to the value for a free neutron (Eq.\,\ref{eq:nEDM_axion}), as a consequence of the Schiff screening theorem for neutral atoms~\cite{Schiff1963}.
The amplitude of the axion DM field, $a_0$, is fixed by the relation $\rho_a \approx m_a^2 a_0^2 /2$.
In this work is it assumed that axions saturate the local cold DM energy density $\rho_{\mathrm{DM}}^{\mathrm{local}} \approx \SI[per-mode=symbol]{0.4}{\giga\electronvolt\per\centi\metre\cubed}$~\cite{Catena2010}.

The derivative coupling of an oscillating galactic axion DM field, $a = a_0 \cos(m_a t - \vtr{p}_a \cdot \vtr{r})$, with spin-polarized nucleons in (\ref{Axion_couplings}) induces time-dependent energy shifts according to:
\begin{equation}
\label{potential_axion-wind}
H_{\textrm{int}} (t) = \frac{C_N a_0}{2 f_a} \sin(m_a t) ~ \vtr{\sigma}_N \cdot \vtr{p}_a \, .
\end{equation}
The term $\vtr{\sigma}_N \cdot \vtr{p}_a$ is conveniently expressed by transforming to a non-rotating celestial coordinate system (see, e.g.,~\cite{Kostelecky1999}):
\begin{align}
\label{sigma-p_a_2}
\vtr{\sigma}_N \cdot \vtr{p}_a  &= \hat{m}_F f(\sigma_N) m_a |\vtr{v}_a|  \notag \\
& \times \left[\cos(\chi) \sin(\delta) + \sin(\chi) \cos(\delta) \cos(\Omega_{\textrm{sid}} t - \eta) \right] \, ,
\end{align}
where $\chi$ is the angle between Earth's axis of rotation and the spin quantization axis ($\chi = \ang{42.5}$ at the location of the PSI), $\delta \approx - \ang{48}$ and $\eta \approx \ang{138}$ are the declination and right ascension of the galactic axion DM flux relative to the Solar System~\cite{NASA2014web}, $\Omega_{\textrm{sid}} \approx \SI{7.29e-5}{\per\second}$ is the daily sidereal angular frequency, $\hat{m}_F = m_F / F$ is the normalized projection of the total angular momentum onto the quantization axis, and $f(\sigma_N) = +1$ for the free neutron, while $f(\sigma_N) = -1/3$ for the $^{199}$Hg atom in the Schmidt (single-particle) model.

The scalar axion-gluon coupling and the vector axion-nucleon coupling would induce harmonic oscillations in the measurements of the nEDM experiment at PSI\@. In the scope of this analysis the time series of the ratio of precession frequencies of polarised neutrons and ${}^{199}$Hg atoms was tested for statistically significant oscillations. In the next chapter the methodology for this quantitative search is introduced.
